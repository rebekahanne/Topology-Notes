%% 
%% This is file, `topology_definitions.tex',
%% generated with the extract package.
%% 
%% Generated on :  2019/03/28,11:38
%% From source  :  "Topology Notes".tex
%% Using options:  active,generate=topology_definitions,extract-env={definition,algorithm}
%% 
\documentclass[11pt]{article}

%%%%%%%%%%%%%%%%%%%%%%%%%%%%%%%%%%%%%%%%%%%%%%%%%%%%%%%%%%%%%%%%%%%%%%%%%%%%%%%%%%

% Packages

% AMS
\usepackage{amsmath, amssymb, amsthm, amsbsy}
% Geometry
\usepackage{geometry}
% Colors
\usepackage[usenames,dvipsnames]{xcolor}
% Figures
\usepackage{graphicx}
\usepackage{float}
% Multi column lists
\usepackage{multicol}
% Subfigures
\usepackage{caption}
\usepackage{subcaption}
% Caligraphic
\usepackage{mathrsfs}
\usepackage{bbm}
% Bold
\usepackage{bm}
% algos
\usepackage[linesnumbered, lined, ruled]{algorithm2e}
% Spacing
\usepackage{setspace}
% Refs/links
\usepackage[colorlinks=true, citecolor=Blue, linkcolor=blue]{hyperref}
\newcommand\myshade{85}
\colorlet{mylinkcolor}{violet}
\colorlet{mycitecolor}{PineGreen}
\colorlet{myurlcolor}{Aquamarine}

\hypersetup{
  linkcolor  = mylinkcolor!\myshade!black,
  citecolor  = mycitecolor!\myshade!black,
  urlcolor   = myurlcolor!\myshade!black,
  colorlinks = true,
}
% Bibliography
\usepackage{filecontents}
\usepackage{natbib}
% Indent
\usepackage{indentfirst}
% Pretty lists
\usepackage{enumitem}
\setlist[enumerate]{itemsep=2pt,topsep=3pt}
\setlist[itemize]{itemsep=2pt,topsep=3pt}
\setlist[enumerate,1]{label=(\roman*)}

% Code
\usepackage{listings}

% Appendix
\usepackage[toc,page]{appendix}

% Math
\usepackage{mathtools}
\usepackage{xparse}

% Equation numbering
\numberwithin{equation}{section}

% Use more than one optional parameter in a new commands
\usepackage{xargs}
% Todo
\usepackage[colorinlistoftodos,prependcaption,textsize=normalsize]{todonotes}
\newcommandx{\unsure}[2][1=]{\todo[linecolor=red,backgroundcolor=red!25,bordercolor=red,#1]{#2}}
\newcommandx{\change}[2][1=]{\todo[linecolor=blue,backgroundcolor=blue!25,bordercolor=blue,#1]{#2}}
\newcommandx{\info}[2][1=]{\todo[linecolor=OliveGreen,backgroundcolor=OliveGreen!25,bordercolor=OliveGreen,#1]{#2}}
\newcommandx{\improvement}[2][1=]{\todo[linecolor=Plum,backgroundcolor=Plum!25,bordercolor=Plum,#1]{#2}}
\newcommandx{\thiswillnotshow}[2][1=]{\todo[disable,#1]{#2}}

% Framed theorems
\usepackage[framemethod=TikZ]{mdframed}

%%%%%%%%%%%%%%%%%%%%%%%%%%%%%%%%%%%%%%%%%%%%%%%%%%%%%%%%%%%%%%%%%%%%%%%%%%%%%%%%%%

% Document Settings

% Figure path
\graphicspath{{./figures/}}
% Matrix columns
\setcounter{MaxMatrixCols}{10}
% So pages will break inside long equation environments
\allowdisplaybreaks
% Font
\usepackage{mathpazo}
\linespread{1.05}
%\usepackage{courier}
% Geometry
\geometry{left=1in,right=1in,top=1in,bottom=1in}
% Counters
\setcounter{tocdepth}{2}
\setcounter{secnumdepth}{3}

%%%%%%%%%%%%%%%%%%%%%%%%%%%%%%%%%%%%%%%%%%%%%%%%%%%%%%%%%%%%%%%%%%%%%%%%%%%%%%%%%%

% Colors

\definecolor{Tm}{rgb}{0,0,0.80}
\newcommand{\navy}[1]{\textcolor{MidnightBlue}{\bf #1}}

%%%%%%%%%%%%%%%%%%%%%%%%%%%%%%%%%%%%%%%%%%%%%%%%%%%%%%%%%%%%%%%%%%%%%%%%%%%%%%%%%%

\newcounter{theo}[section]\setcounter{theo}{0}
\renewcommand{\thetheo}{\arabic{section}.\arabic{theo}}

\newenvironment{theo}[2][]{%
    \refstepcounter{theo}

\ifstrempty{#1}%
% if condition (without title)
{\mdfsetup{%
    frametitle={%
        \tikz[baseline=(current bounding box.east),outer sep=0pt]
        \node[anchor=east,rectangle,fill=blue!20]
        {\strut Theorem~\thetheo};}
    }%
% else condition (with title)
}{\mdfsetup{%
    frametitle={%
        \tikz[baseline=(current bounding box.east),outer sep=0pt]
        \node[anchor=east,rectangle,fill=blue!20]
        {\strut Theorem~\thetheo:~#1};}%
    }%
}%
% Both conditions
\mdfsetup{%
    innertopmargin=10pt,linecolor=blue!20,%
    linewidth=2pt,topline=true,%
    frametitleaboveskip=\dimexpr-\ht\strutbox\relax%
}

\begin{mdframed}[]\relax}{%
\end{mdframed}}

\newcounter{prf}[section]\setcounter{prf}{0}
\renewcommand{\theprf}{\arabic{section}.\arabic{prf}}
\newenvironment{prf}[2][]{%
\refstepcounter{prf}%
\ifstrempty{#1}%
{\mdfsetup{%
frametitle={%
\tikz[baseline=(current bounding box.east),outer sep=0pt]
\node[anchor=east,rectangle,fill=red!20]
{\strut Proof~\theprf};}}
}%
{\mdfsetup{%
frametitle={%
\tikz[baseline=(current bounding box.east),outer sep=0pt]
\node[anchor=east,rectangle,fill=red!20]
{\strut Proof~\thetheo:~#1};}}%
}%
\mdfsetup{innertopmargin=10pt,linecolor=red!20,%
linewidth=2pt,topline=true,%
frametitleaboveskip=\dimexpr-\ht\strutbox\relax
}
\begin{mdframed}[]\relax%
\label{#2}}{\qed\end{mdframed}}

% Environments

\theoremstyle{definition}
\newmdtheoremenv{theorem}{\color{ForestGreen}{\textbf{Theorem}}}[section]
% \newtheorem{theorem}{\color{ForestGreen}{\textbf{Theorem}}}[section]
\newtheorem{claim}{\color{ForestGreen}{\textbf{Claim}}}[section]
% \newtheorem{lemma}[theorem]{\color{ForestGreen}{\textbf{Lemma}}}
% \newtheorem{proposition}[theorem]{\color{ForestGreen}{\textbf{Proposition}}}
% \newtheorem{corollary}[theorem]{\color{ForestGreen}{\textbf{Corollary}}}
\newmdtheoremenv{lemma}[theorem]{\color{ForestGreen}{\textbf{Lemma}}}
\newmdtheoremenv{proposition}[theorem]{\color{ForestGreen}{\textbf{Proposition}}}
\newmdtheoremenv{corollary}[theorem]{\color{ForestGreen}{\textbf{Corollary}}}

\newtheorem{axiom}[theorem]{\color{ForestGreen}{\textbf{Axiom}}}
\newtheorem{conjecture}[theorem]{Conjecture}
\newtheorem{case}[theorem]{Case}
\newtheorem{conclusion}[theorem]{Conclusion}
\newtheorem{criterion}[theorem]{Criterion}
\newtheorem{notation}[theorem]{Notation}
\newtheorem{problem}[theorem]{Problem}

\theoremstyle{definition}
% \newtheorem{definition}{\color{MidnightBlue}{\textbf{Definition}}}[section]
\newmdtheoremenv{definition}{\color{MidnightBlue}{\textbf{Definition}}}[section]
\newtheorem{example}{\color{WildStrawberry}Example}[section]
\newtheorem{assumption}{Assumption}[section]
\newtheorem{condition}[assumption]{Condition}
\newtheorem*{solution}{\color{Goldenrod}Solution}
% \newenvironment{solution}[1][\proofname]{%
%   \proof[\bf \color{Goldenrod}Solution to #1]%
% }{\endproof}

\newtheorem{exercise}{\color{YellowOrange}Exercise}[section]

% Literature Summary Standards
\newtheorem*{motivation}{Motivation}
\newtheorem*{summary}{Summary}
\newtheorem*{remark}{Remark}
\newtheorem*{model}{Model}
\newtheorem*{tresults}{Theoretical Results}
\newtheorem*{eresults}{Empirical Results}

%%%%%%%%%%%%%%%%%%%%%%%%%%%%%%%%%%%%%%%%%%%%%%%%%%%%%%%%%%%%%%%%%%%%%%%%%%%%%%%%%%

% Math macros

% Math ``brackets''
\newcommand\parens[1]{\left( #1 \right)}
\newcommand\squares[1]{\left[ #1 \right]}
\newcommand\braces[1]{\left\{ #1 \right\}}
\newcommand\angles[1]{\left\langle #1 \right\rangle}
\newcommand\ceil[1]{\left\lceil #1 \right\rceil}
\newcommand\floor[1]{\left\lfloor #1 \right\rfloor}
\newcommand\abs[1]{\left| #1 \right|}
\newcommand\dabs[1]{\left\| #1 \right\|}
\newcommand\vect[1]{\mathbf{#1}}
\newcommand\closure[1]{\overline{#1}}
\newcommand\pset[1]{\mathcal{P}\left(#1\right)}
\newcommand\inv[1]{#1^{-1}}
\newcommand\norm[1]{\lVert#1\rVert}

% inner product
\providecommand{\inner}[1]{\left\langle{#1}\right\rangle}
% stochastic dominance
\newcommand{\lesd}{\preceq_{\textrm{SD}}}

% Set builder (use \Set ultimately and separate by ;)
\DeclarePairedDelimiterX{\set}[1]{\{}{\}}{\setargs{#1}}
\NewDocumentCommand{\setargs}{>{\SplitArgument{1}{;}}m}
{\setargsaux#1}
\NewDocumentCommand{\setargsaux}{mm}
{\IfNoValueTF{#2}{#1} {#1\nonscript\:\delimsize\vert\allowbreak\nonscript\:\mathopen{}#2}}%
\def\Set{\set*}%

% Shortcut math
\newcommand{\ls}{\leqslant}
\newcommand{\gs}{\geqslant}
\def\ss{\subset}
\def\sse{\subseteq}
\def\nss{\not \ss}
\def\sps{\supset}
\def\pss{\subsetneq}
\def\prece{\preccurlyeq}
\def\condgap{\hspace{1cm}}
\def\eprec{\preceq}
% argmax and min
\newcommand{\argmax}{\operatornamewithlimits{argmax}}
\newcommand{\argmin}{\operatornamewithlimits{argmin}}
\newcommand{\es}{\emptyset}
% Implication and reverse implication
\def\imp{\Rightarrow}
\def\pmi{\Leftarrow}
% Integers up to number
\newcommand\intsfin[1]{\braces{1, \ldots, #1}}
% Logic
\def\bic{\Leftrightarrow}
% Bold and italic
\newcommand\boldit[1]{\textbf{\textit{#1}}}
% Misc math
\newcommand{\st}{\ensuremath{\ \mathrm{s.t.}\ }}
\newcommand{\setntn}[2]{ \{ #1 : #2 \} }
\newcommand{\cf}[1]{ \lstinline|#1| }
\newcommand{\fore}{\therefore \quad}
\newcommand{\tod}{\stackrel { d } {\to} }
\newcommand{\tow}{\stackrel { w } {\to} }
\newcommand{\toprob}{\stackrel { p } {\to} }
\newcommand{\toms}{\stackrel { ms } {\to} }
\newcommand{\eqdist}{\stackrel{d} {=} }
\newcommand{\iidsim}{\stackrel{\textrm{ {\sc iid }}} {\sim} }
\newcommand{\1}{\mathbbm 1}
\newcommand{\dee}{\,{\rm d}}
\newcommand{\given}{\, | \,}
\newcommand{\la}{\langle}
\newcommand{\ra}{\rangle}

% Shortcut greek
\def\a{\alpha}
\def\b{\beta}
\def\g{\gamma}
\def\D{\Delta}
\def\d{\delta}
\def\z{\zeta}
\def\k{\kappa}
\def\l{\lambda}
\def\n{\nu}
\def\r{\rho}
\def\s{\sigma}
\def\t{\tau}
\def\x{\xi}
\def\w{\omega}
\def\W{\Omega}
% Nice greek
\newcommand{\p}{\varphi}
\newcommand{\e}{\varepsilon}

% Shorcut vectors
\def\vx{\vect{x}}
\def\vy{\vect{y}}
\def\va{\vect{a}}
\def\vb{\vect{b}}

\newcommand{\CC}{\mathbb C}
\newcommand{\FF}{\mathbb F}
\newcommand{\RR}{\mathbb R}
\newcommand{\NN}{\mathbb N}
\newcommand{\PP}{\mathbbm P}
\newcommand{\EE}{\mathbbm E}
\newcommand{\TT}{\mathbbm T}
\newcommand{\VV}{\mathbbm V}
\newcommand{\QQ}{\mathbb Q}
\newcommand{\WW}{\mathbbm W}
\newcommand{\ZZ}{\mathbbm Z}
\newcommand{\UU}{\mathbbm U}
\renewcommand{\SS}{\mathbbm S}

% Expectation/Probability
\newcommand{\ee}[1]{\mathbbm{E}[{#1}]}
\newcommand{\pp}[1]{\mathbbm{P}({#1})}

\newcommand{\GG}{\mathsf G}
\newcommand{\XX}{\mathsf X}
\renewcommand{\AA}{\mathsf A}
\newcommand{\YY}{\mathsf Y}
\newcommand{\ZZZ}{\mathsf Z}

\newcommand{\aA}{\mathscr A}
\newcommand{\iI}{\mathscr I}
\newcommand{\eE}{\mathscr E}
\newcommand{\rR}{\mathscr R}
\newcommand{\lL}{\mathscr L}
\newcommand{\cG}{\mathscr G}

\newcommand{\pP}{\mathcal P}
\newcommand{\aAA}{\mathcal A}
\newcommand{\vV}{\mathcal V}
\newcommand{\mM}{\mathcal M}
\newcommand{\oO}{\mathcal O}
\newcommand{\gG}{\mathcal G}
\newcommand{\hH}{\mathcal H}
\newcommand{\tT}{\mathcal T}
\newcommand{\bB}{\mathcal B}
\newcommand{\zZ}{\mathcal Z}
\newcommand{\cC}{\mathcal C}
\newcommand{\dD}{\mathcal D}
\newcommand{\wW}{\mathcal W}
\newcommand{\uU}{\mathcal U}
\newcommand{\sS}{\mathcal S}
\newcommand{\fF}{\mathcal F}

% Common collections
\def\cA{\col{A}}
\def\cB{\col{B}}
% \def\cC{\col{C}}
\def\cT{\col{T}}
\def\cU{\col{U}}

% Common closures
\def\clA{\closure{A}}
\def\clB{\closure{B}}
\def\clK{\closure{K}}

% operators
\DeclareMathOperator{\cl}{cl}
\DeclareMathOperator{\graph}{graph}
\DeclareMathOperator{\interior}{int}
\DeclareMathOperator{\Prob}{Prob}
\DeclareMathOperator{\determinant}{det}
\DeclareMathOperator{\trace}{trace}
\DeclareMathOperator{\sgn}{sgn}
\DeclareMathOperator{\Span}{span}
\DeclareMathOperator{\diag}{diag}
\DeclareMathOperator{\proj}{proj}
\DeclareMathOperator{\rank}{rank}
\DeclareMathOperator{\cov}{Cov}
\DeclareMathOperator{\corr}{Corr}
\DeclareMathOperator{\var}{Var}
\DeclareMathOperator{\mse}{mse}
\DeclareMathOperator{\se}{se}
\DeclareMathOperator{\row}{row}
\DeclareMathOperator{\col}{col}
\DeclareMathOperator{\range}{rng}
\DeclareMathOperator{\kernel}{ker}
\DeclareMathOperator{\dimension}{dim}
\DeclareMathOperator{\bias}{bias}
\DeclareMathOperator{\dom}{dom}
\DeclareMathOperator{\ran}{ran}
\DeclareMathOperator{\Int}{Int}
\DeclareMathOperator{\Cl}{Cl}
\DeclareMathOperator{\im}{im}


\begin{document}

\begin{definition}[Set cardinality $\leq$]
Let $A, B$ be sets. $A$ has \navy{cardinality less than or equal to} $B$ (write $\abs{A} \leq \abs{B}$) if there exists an injection from $A$ to $B$. In notation,
\begin{equation}
\abs{A} \leq \abs{B} \iff \exists f:A \to B \text{ injective }
\end{equation}
\end{definition}

\begin{definition}[Partial order]
A \navy{partial order} is a pair $\aAA = (A, \triangleright)$ where $A \neq \emptyset$ such that  for $a,b,c \in A$
\begin{enumerate}
\item Antireflexivity: $a \triangleright a$ never happens.
\item Transitivity: $a \triangleright b, b \triangleright c \imp a \triangleright c$
\end{enumerate}
\end{definition}

\begin{definition}[Maximal]
Let $(A, \triangleright)$ be a partial order. Then $m \in A$ is maximal if and only if no $a \triangleright m$.
\end{definition}

\begin{definition}[Chain]
A \navy{chain} in a partial order $(A, \triangleright)$ is a $C \subseteq A$ such that $\forall a,b \in C$, $a=b$ or $a \triangleright b$ or $b \triangleright a$. (One interpretation in words, ``$C$ is linear'')
\end{definition}

\begin{definition}[Topology, Topological Space]
A \navy{topological space} is a pair $(X,\tT)$ where $X$ is a nonempty set and $\tT$ is a set of subsets of $X$ (called a \navy{topology}) having the following properties:
\begin{enumerate}
\item $\emptyset$ and $X$ are in $\tT$.
\begin{equation}
\emptyset \in \tT,\ X \in \tT
\end{equation}
\item The union of \textit{arbitrarily} many sets in $\tT$ is in $\tT$.
\begin{equation}
A_\eta \in \tT \ \forall \eta \in H \imp \bigcup_{\eta \in H} A_\eta = \Set{a \in X \text{ for some (that is, } a \in A_\eta \text{)}} \in \tT
\end{equation}
\item The intersection a \textit{finite} number of sets in $\tT$ is in $\tT$.
\begin{equation}
A_1, \ldots, A_n \in \tT \imp A_1 \cap \cdots \cap A_n \in \tT
\end{equation}
\end{enumerate}
A set $X$ for which a topology $\tT$ has been specified is called a \navy{topological space}, that is, the pair $(X,\tT)$.
\end{definition}

\begin{definition}[(Strictly) Finer, (Strictly) Coarser, Comparable]
Suppose $\tT$ and $\tT'$ are two topologies on a given set $X$. If $\tT' \supset \tT$, we say that $\tT'$ is \navy{finer} than $\tT$. If $\tT'$ properly contains $\tT$, we say that $\tT'$ is \navy{strictly finer} than $\tT$. For these respective situations, we say that $\tT$ is \navy{coarser} or \navy{strictly coarser} than $\tT'$. We say that $\tT$ is \navy{comparable} with $\tT'$ if either $\tT' \supset \tT$ or $\tT \supset \tT'$.
\end{definition}

\begin{definition}[Basis, Basis Elements, Topology $\tT$ generated by $\bB$]
If $X$ is a set, a \navy{basis} for a topology on $X$ is a collection $\bB \ss \pP(X)$ of subsets of $X$ (called \navy{basis elements}) such that
\begin{enumerate}
\item Every element $x \in X$ belongs to some set in $\bB$. In symbols
\begin{equation}
\forall x \in X \ \exists B \in \bB \ s.t. \ x \in B
\end{equation}
\item If $x$ belongs to the intersection of two basis elements $B_1$ and $B_2$, then there is a basis element $B_3$ containing $x$ such that $B_3 \subset B_1 \cap B_2$. More generally, in symbols
\begin{equation}
\forall B_1,\ldots,B_n \in \bB \ \forall x \in \bigcap_{i=1}^n B_i \ \exists B \in \bB \ s.t. \ x \in B \ss \bigcap_{i=1}^n B_i
\end{equation}
\end{enumerate}
\end{definition}

\begin{definition}[Topology $\tT$ generated by $\bB$]
If $\bB$ is a basis for a topology on $X$, then we define the \navy{topology $\tT$ generated by $\bB$} as follows: A subset $U$ of $X$ is said to be open in $X$ (that is, to be an element of $\tT$) if for each $x \in U$, there is a basis element $B_x \in \bB$ such that $x \in B_x$ and $B_x \subset U$. Note that each basis element is itself an element of $\tT$. More succinctly:
\begin{equation*}
U \ss X: U \in \tT \iff \forall x\in U \ \exists B_x \in \bB \ s.t. \ x \in B_x \ss U
\end{equation*}

\end{definition}

\begin{definition}[Subbasis]
A \navy{subbasis} for a topology on $X$ is a collection of subsets $\sS \ss \pP(X)$ of $X$ whose union equals $X$ (that is, $\bigcup \sS = X$). The topology generated by the subbasis $\sS$ is defined to be the collection $\tT$ of all unions of finite intersections of elements of $\sS$.
\end{definition}

\begin{definition}[Closed]
A subset $A$ of a topological space $(X,\tT)$ is said to be \navy{closed} if the set $X - A \in \tT$. In words, a subset of a topological space is closed if its complement (in the space) is open.
\end{definition}

\begin{definition}[Continuous]
Let $(X,\tT)$ and $(Y,\sigma)$ be topological spaces. A function $f : X \to Y$ is said to be \navy{continuous} with respect to $\tT$ and $\sigma$ if for each open subset $V$ of $Y$, the set $f^{-1}(V)$ is an open subset of $X$. In symbols, $\forall S \in \sigma$, we have that $\inv{f}(S) \in \tT$ (where $\inv{f}(S) = \Set{a \in X; f(a) \in S}$). In words, the preimage of an open set is open.
\end{definition}

\begin{definition}[Homeomorphic]
The topological spaces $(X,\tT)$ and $(Y,\sigma)$ are \navy{homeomorphic} if there exists a function $f: X \to Y$ such that
\begin{enumerate}
\item $f$ is bijective.
\item $f$ is continuous.
\item $\inv{f}$ is also continuous.
\end{enumerate}
We write $(X,\tT) \cong (Y,\sigma)$ or $f: (X,\tT) \cong (Y,\sigma)$.
\end{definition}

\begin{definition}[Open map]
$f: (X,\tT) \to (Y,\sigma)$ is an \navy{open map} if $\forall S \in \tT$ we have that $f(S) \in \sigma$ (recall $f(S) = \Set{f(s);s \in S}$). In words: open sets map to open sets.
\end{definition}

\begin{definition}[Closed map]
$f: (X,\tT) \to (Y,\sigma)$ is a \navy{closed map} if $\forall S \ss X$ such that $X - S \in \tT$ we have that $Y - f(S) \in \sigma$. In words: closed sets map to closed sets.
\end{definition}

\begin{definition}[Subspace topology]
Given a topological space $(X,\tT)$ and a non-empty set $A \sse X$, the \navy{subspace topology on $A$ induced (or given) by $\tT$} is $(A, \t_A)$ where $\tT_A = \Set{U \cap A; U \in \tT}$.
\end{definition}

\begin{definition}[Connected]
A space $(X,\t)$ is \navy{connected} if whenever sets $V,W$ are
\begin{enumerate}
\item Nonempty
\item Open
\item $V \cup W = X$
\end{enumerate}
we have $V \cap W \neq \emptyset$.
\end{definition}

\begin{definition}[Metric space]
A \navy{metric space} is a nonempty set $X$ together with a binary function $d : X\times X \to \RR$ which satisfies the following properties: For all $x,y,z \in X$ we have that
\begin{enumerate}
\item Positivity: $d(x,y) \geq 0$
\item Definiteness: $d(x,y) = 0$ if and only if $x=y$
\item Symmetry: $d(x,y) = d(y,x)$
\item Triangle inequality: $d(x,y) + d(y,z) \geq d(x,z)$
\end{enumerate}
\end{definition}

\begin{definition}[Metric topology]
Given a metric space $(X,d)$, the set $\bB = \Set{B_\e(x); \e > 0, x \in X}$ is a basis for a topology on $X$ (where $B_\e(x) = \Set{y \in X; d(x,y) < \e}$) called the \navy{metric topology}.
\end{definition}

\begin{definition}[Open Set in Metric Topology]
A set is \navy{open} in the metric topology induced by $d$ if and only if for each $y \in U$ there is a $\d > 0$ such that $B_d(y, \d) \ss U$.
\end{definition}

\begin{definition}[$T_2$, Hausdorff]
$(X,\t)$ is \navy{$T_2$ (Hausdorff)} if for every distinct $a,b \in X$, there exist open sets $V,W \in \t$ such that $a \in V$, $b \in W$, and $V \cap W = \emptyset$.
\end{definition}

\begin{definition}[Converges]
Fix a topological space $(X,\tT)$. A sequence of points $(a_i)_{i \in \NN} \ss X$ \navy{converges} to $b \in X$ if for every open set $W$ containing $b$, all but finitely many of the terms of the sequence are in $W$. In symbols
\begin{equation*}
(a_i)_{i \in \NN} \to b \iff \forall W\in \tT \text{ s.t. } b \in W, \exists N \in \NN \text{ s.t. } \forall m > n, a_m \in W
\end{equation*}

\end{definition}

\begin{definition}[Sequentially closed]
A set $S \ss X$ is \navy{sequentially closed} if for every sequence $(a_i)_{i \in \NN}$ of points in $S$ converging to some $b \in X$, we have $b \in S$.
\end{definition}

\begin{definition}[Product topology, two sets]
Let $(X,\t)$ and $(Y,\sigma)$ be topological spaces. The \navy{product topology} on $X \times Y$ is the topology having as \emph{basis} the collection $\bB$ of all set of the form $U \times V$ where $U$ is an open subset of $X$ and $V$ is an open subset of $Y$. In symbols,
\begin{equation}
\bB_{\t \times \sigma} = \Set{U \times V; U \in \t, V \in \sigma}
\end{equation}
\end{definition}

\begin{definition}[Product space/topology for finitely many spaces]
Let $(X_1,\t_1), \ldots, (X_n,\t_n)$ be topological spaces. The set of points of the \navy{product space} is $X_1 \times \cdots \times X_n$. The basis for the \navy{product topology} is
\begin{equation}
\bB_{\t_1 \times \cdots \times \t_n} = \Set{W_1 \times \cdots \times W_n; W_i \in \t_i}
\end{equation}
\end{definition}

\begin{definition}[Projection Maps]
Let $(X_1,\t_1)$ and $(X_2,\t_2)$ be topological spaces. Let
\begin{align*}
&\pi_1:X_1 \times X_2 \to X_1 : (x_1,x_2) \mapsto x_1 \\
&\pi_2:X_1 \times X_2 \to X_2 : (x_1,x_2) \mapsto x_2
\end{align*}
then $\pi_1$ and $\pi_2$ are \navy{projection maps}.
\end{definition}

\begin{definition}[Open cover]
An \navy{open cover} of $(X,\t)$ is a family of $\t$-open sets $\cC \ss \t$ such that $\bigcup \cC = X$.
\end{definition}

\begin{definition}[Subcover]
$\dD$ is a \navy{subcover} of $\cC$ if
\begin{enumerate}
\item $\dD \ss \cC$.
\item $\dD$ is an open cover.
\end{enumerate}
A \navy{finite subcover} is a subcover which is finite.
\end{definition}

\begin{definition}[Compact]
A topological space $(X,\t)$ is \navy{compact} if every open cover has a finite subcover.
\end{definition}

\begin{definition}[Cartesian Product (Possibly Infinite)]
Let $\Set{X_i}_{i\in I}$ be a family of sets, where $I$ is an arbitrary index set. The \navy{Cartesian product} $\prod_{i \in I} X_i$ is the set of all functions $f$ such that
\begin{enumerate}
\item $f: I \to \cup_{i \in I} X_i$
\item $f(i) \in X_i$
\end{enumerate}
In words, $f$ picks out a point from each set. In notation
\begin{equation}
\prod_{i \in I} X_i = \Set{f: I \to \cup_{i \in I} X_i; f(i) \in X_i) \forall i \in I}
\end{equation}
\end{definition}

\begin{definition}[Product Space]
Let $I$ be an arbitrary index set. Suppose for each $i \in I$ we have that $(X_i, \t_i)$ is a topological space. Their \navy{product space} has
\begin{itemize}
\item Underlying set:
\begin{equation}
\prod_{i \in I} X_i
\end{equation}
\item Topology:
\begin{itemize}
\item Informally: $\bigotimes_{i \in I} \t_i$ generated by the subbasis of ``wedges'': that is, the topology generated by the subbasis
\begin{equation}
\bB = \text{``all sets of the form''} \quad \cdots \times X_i \times X_j \times \cdots \times \underbrace{U}_{n \text{th term}} \times X_k \times \cdots
\end{equation}
for $U$ open in $\t_n$.
\improvement[inline]{Figure.}
\item Formally: $\bigotimes_{i \in I} \t_i$ generated by the subbasis $\bB$, which is sets of the form
\begin{equation}
\Set{f \in \prod_{i \in I} X_i; f(k) \in W} \quad k \in I, W \in \t_k
\end{equation}
\item Alternative: $\bigotimes_{i \in I} \t_i$ is the coarsest topology making all projection maps continuous (where $\pi_i: f \mapsto f(i)$).
\begin{equation}
\pi_i^{-1}(U) \in \bB
\end{equation}
\improvement[inline]{What does this mean?}
\end{itemize}
\end{itemize}
\end{definition}

\begin{definition}[Ultrafilter]
Suppose $I$ is an set (wlog infinite). An \navy{ultrafilter $\UU$} on $I$ is a family of subsets of $I$ such that:
$\UU$ is a filter:
\begin{enumerate}
\item \textbf{Contains $I$, not $\emptyset$}: $I \in \UU$, $\emptyset \not\in \UU$.
\item \textbf{Closed upwards}: $A \in \UU, \ A \sse B \imp B \in \UU$.
\item \textbf{Closed under finite intersections:} $A_1, \ldots, A_n \in \UU \imp A_1 \cap \cdots \cap A_n \in \UU$.
\end{enumerate}
and $\UU$ satisfies the additional ``ultra'' condition:
\begin{enumerate}
\item[(iv)] $\forall A \sse I$, $A \in \UU$ or $I - A \in \UU$ (but not both, by (i) and (iii))
\end{enumerate}
\end{definition}

\begin{definition}[Principal ultrafilter]
\navy{Principal ultrafilters} take the form
\begin{equation}
\angles{a} = \Set{A \sse I: a \in A} \qquad \text{ for some } a \in I
\end{equation}
\end{definition}

\begin{definition}[Finite intersection property (FIP)]
A collection of subsets $\fF$ of $I$ has \navy{FIP} if whenever $A_1,\ldots,A_n \in \fF$ we have $\bigcap_{i=1}^n A_i \neq \emptyset$.
\end{definition}

\begin{definition}[Ultrafilter convergence]
Suppose $(X,\t)$ is a topological space and $\UU$ is an ultrafilter on $X$. Then $\UU$ \navy{converges} to $\alpha$ (and we write $\UU \to \alpha$) for $\alpha \in X$ if every open set containing $\alpha$ is in $\UU$. In symbols
\begin{equation}
\UU \to \alpha \iff \forall V_\alpha \in \t : \alpha \in V_\alpha, V_\alpha \in \UU
\end{equation}
\end{definition}

\begin{definition}[Quotient Map]
Let $(X,\t)$, $(Y,\sigma)$ be topological spaces. A map $f: (X,\t) \to (Y,\sigma)$ is a \navy{quotient map} with respect to $(\t,\sigma)$ iff
\begin{enumerate}
\item $\forall A \sse Y: A \in \sigma \iff \inv{f}(A) \in \t$ (in words, a subset $A$ of $Y$ is open in $Y$ iff $\inv{f}(A)$ is open in $X$).
\item $f$ surjective
\end{enumerate}
\end{definition}

\begin{definition}[Quotient topology]
If $(X,\t)$ is a topological space and $f:(X,\t) \to Y$ ($Y \neq \emptyset$) surjective, then the \navy{quotient topology} given by $f$ is
\begin{equation}
\sigma = \Set{A \sse Y; \inv{f}(A) \in \t}
\end{equation}
\end{definition}

\begin{definition}[Difference]
The \navy{difference} of two sets, denoted $A-B$, is the set consisting of those elements of $A$ that are not in $B$. In notation
\begin{equation*}
A - B = \{x | x \in A \text{ and } x \not\in B\}
\end{equation*}
\end{definition}

\end{document}
